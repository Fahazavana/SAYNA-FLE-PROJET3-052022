\documentclass[12pt]{article}
\usepackage[margin=10mm]{geometry}
\usepackage[T1]{fontenc}
\usepackage[utf8]{inputenc}
\usepackage[french]{babel}
\usepackage{array}
\usepackage{hyperref}
\usepackage{pifont}
\usepackage{colortbl}
\usepackage{color}
\newcommand{\und}[1]{\underline{#1}}
\newcommand{\colo}[1]{{\color{blue}\textbf{#1}}}

\author{Jean Lucien RANDRIANANTENANA}
\title{Vers un français
impeccable (B2)\\SAYNA-FLE-PROJET1-052022}
\begin{document}
\maketitle
\tableofcontents
\newpage
\section{Substituer au mot souligné un terme plus précis}
\begin{enumerate}
	 \item Le travail est une chose biologiquement nécessaire. \item L’employé a oublié de servir les gens. \item Elle s’est présentée après l’heure convenue. Cela a indisposé le patron. \item Elle veut devenir une chanteuse célèbre. C’est louable. \item Dans son regard, il y avait de l’intelligence. \item A la piscine municipale, il y a des baigneurs.
\end{enumerate}

\section{Remplacez le verbe mettre par un verbe plus précis} 
\begin{enumerate}
	\item Par amitié, il a mis la main sur son épaule. \item Il a mis toutes ses «économies dans son prêt. \item Elle a mis ses plus beaux vêtements pour cette occasion. \item J’ai faim, mets un peu de gâteau dans mon assiette. \item Ne laisse pas traîner tes jouets, mets-les dans le coffre. \item Cette feuille ne m’est d,aucune utilité, mets-la aux ordures. \item Prends ce colis et mets-le chez la voisine s’il-te-plaît. \item Il a mis la clé dans la serrure et a déverrouillé le cadenas. \item Il met trop de temps à faire ses devoirs.
\end{enumerate}
\section{Remplacez le verbe faire par un verbe plus précis} 
\begin{enumerate}
	\item L’écrivain Victor Hugo a fait une œuvre considérable. \item Cette année, j’ai fait l’Andalousie. \item La DRH a fait une lettre de recommandation pour son stagiaire. \item Le chien a fait six kilomètres pour revenir jusqu’à la maison. \item Mon nouveau voisin fait du bruit la nuit : je ne peux m’y faire. \item La personne qui ment fait une faute grave. \item Il a vendu ses anciens vêtements pour se faire un peu d’argent. \item Le nouveau-né faisait 3,3 kg à sa naissance. \item Il a fait le tour de la piste en moins d’une minute. \item Pendant le cours, l’élève distrait faisait des dessins sur la table.
\end{enumerate}

\section{Remplacez le verbe avoir par un verbe plus précis}
\begin{enumerate}
	\item Venez dans notre boutique; nos clients auront aujourd’hui une réduction spéciale. \item Il n’est pas allé à son cours; il a mal au dos. \item Dans son milieu professionnel, il a une excellente réputation. \item J’ai une grosse angoisse. \item Chaque question a sa réponse. \item Ce magasin a des produits venu de la côte. \item Hier, Yasmine a ses nouvelles chaussures. \item Le patient a une rage de dent. \item Elle a trop de rêves en tête, elle ferait mieux de se concentrer sur ses études ! 1\item Le candidat ne s’est pas présenté à son entretien : il a eu des difficultés sur la route.
\end{enumerate}
\end{document}