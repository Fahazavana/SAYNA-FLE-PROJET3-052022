\documentclass[12pt]{article}
\usepackage[margin=10mm]{geometry}
\usepackage[T1]{fontenc}
\usepackage[utf8]{inputenc}
\usepackage[french]{babel}
\usepackage{array}
\usepackage{hyperref}
\usepackage{pifont}
\usepackage{colortbl}
\usepackage{color}
\newcommand{\und}[1]{\underline{#1}}
\newcommand{\colo}[1]{{\color{blue}\textbf{#1}}}

\author{Jean Lucien RANDRIANANTENANA}
\title{Français Langue
Étrangère - Perfectionnement\\SAYNA-FLE-PROJET3-052022}
\begin{document}
\maketitle
\tableofcontents
\newpage
\section{Substituer au mot souligné un terme plus précis}
\begin{enumerate}
	 \item Le travail est une chose biologiquement nécessaire. \item L’employé a oublié de servir les gens. \item Elle s’est présentée après l’heure convenue. Cela a indisposé le patron. \item Elle veut devenir une chanteuse célèbre. C’est louable. \item Dans son regard, il y avait de l’intelligence. \item A la piscine municipale, il y a des baigneurs.
\end{enumerate}

\section{Remplacez le verbe mettre par un verbe plus précis} 
\begin{enumerate}
	\item Par amitié, il a \colo{posé} sa main sur son épaule. 
	\item Il a mis toutes ses \og{}économies dans son prêt\fg{}. 
	\item Elle a \colo{porté} ses plus beaux vêtements pour cette occasion. 
	\item J’ai faim, mets un peu de gâteau dans mon assiette.
	\item Ne laisse pas traîner tes jouets, \colo{range}-les dans le coffre. 
	\item Cette feuille ne m’est d,aucune utilité, \colo{jette}-la aux ordures. 
	\item Prends ce colis et \colo{donne}-le chez la voisine s’il-te-plaît. 
	\item Il a \colo{entré} la clé dans la serrure et a déverrouillé le cadenas. 
	\item Il \colo{prend} trop de temps à faire ses devoirs.
\end{enumerate}
\section{Remplacez le verbe faire par un verbe plus précis} 
\begin{enumerate}
	\item L’écrivain Victor Hugo a \colo{écrit} une œuvre considérable. 
	\item Cette année, j’ai \colo{visité} l’Andalousie. 
	\item La DRH a \colo{écrit} une lettre de recommandation pour son stagiaire. 
	\item Le chien a \colo{parcouru} six kilomètres pour revenir jusqu’à la maison. 
	\item Mon nouveau voisin \colo{est bruyant} la nuit : je ne peux m’y faire. 
	\item La personne qui ment \colo{commet} une faute grave. 
	\item Il a vendu ses anciens vêtements pour \colo{gagner} un peu d’argent. 
	\item Le nouveau-né \colo{pesait} 3,3 kg à sa naissance. 
	\item Il a \colo{fini} le tour de la piste en moins d’une minute. 
	\item Pendant le cours, l’élève distrait \colo{dessinait} sur la table.
\end{enumerate}

\section{Remplacez le verbe avoir par un verbe plus précis}
\begin{enumerate}
	\item Venez dans notre boutique; nos clients \colo{obtiendront}  une réduction spéciale aujourd’hui. 
	\item Il n’est pas allé à son cours; il \colo{souffre} au dos. 
	\item Dans son milieu professionnel, il a une excellente réputation. 
	\item Je \colo{suis} trop \colo{angoissé}. 
	\item Chaque question \colo{possède} sa réponse. 
	\item Ce magasin \colo{vend} des produits venu de la côte. 
	\item Hier, Yasmine \colo{porte} ses nouvelles chaussures. 
	\item Le patient \colo{souffre d'} une rage de dent. 
	\item Elle \colo{est} trop \colo{distrait}, elle ferait mieux de se concentrer sur ses études ! 
	\item Le candidat ne s’est pas présenté à son entretien : il a \colo{rencontré} des difficultés sur la route.
\end{enumerate}
\end{document}